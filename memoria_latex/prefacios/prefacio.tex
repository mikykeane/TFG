\chapter*{}
%\thispagestyle{empty}
%\cleardoublepage

%\thispagestyle{empty}

%%\begin{titlepage}
 
 
\setlength{\centeroffset}{-0.5\oddsidemargin}
\addtolength{\centeroffset}{0.5\evensidemargin}
\thispagestyle{empty}

\noindent\hspace*{\centeroffset}\begin{minipage}{\textwidth}

\centering
\includegraphics[width=0.9\textwidth]{imagenes/logo_ugr.jpg}\\[1.4cm]

\textsc{ \Large PROYECTO FIN DE CARRERA\\[0.2cm]}
\textsc{ INGENIERÍA EN INFORMÁTICA}\\[1cm]
% Upper part of the page
% 

 \vspace{3.3cm}

%si el proyecto tiene logo poner aquí
\includegraphics{imagenes/logo.png} 
 \vspace{0.5cm}

% Title

{\Huge\bfseries Vigilancia Tecnológica y Minería de Opiniones en RRSS\\
}
\noindent\rule[-1ex]{\textwidth}{3pt}\\[3.5ex]
{\large\bfseries Subtítulo del proyecto.\\[4cm]}
\end{minipage}

\vspace{2.5cm}
\noindent\hspace*{\centeroffset}\begin{minipage}{\textwidth}
\centering

\textbf{Autor}\\ {Miguel Keane Cañizares (alumno)}\\[2.5ex]
\textbf{Director}\\
{Antonio Gabriel López Herrera(tutor)}\\[2cm]
%Nombre Apellido1 Apellido2 (tutor2)}\\
\includegraphics[width=0.15\textwidth]{imagenes/tstc.png}\\[0.1cm]
\textsc{Departamento de Ciencias de la Computación e Inteligencia Artificial}\\
%\textsc{---}\\
Granada, Septiembre de 2019
\end{minipage}
\addtolength{\textwidth}{\centeroffset}
\vspace{\stretch{2}}

 
%\end{titlepage}






\cleardoublepage
\thispagestyle{empty}

\begin{center}
{\large\bfseries Vigilancia Tecnológica y Minería de Opiniones en RRSS}\\
\end{center}
\begin{center}
Miguel Keane Cañizares \\
\end{center}

%\vspace{0.7cm}
\noindent{\textbf{Palabras clave}: RRSS, Twitter, Netflix, HBO, Streaming, Meaningclud, Wordcloud, Tweepy, MongoDB, Pymongo, Python}\\

\vspace{0.7cm}
\noindent{\textbf{Resumen}}\\

Hoy en día las redes sociales son la mayor fuente de información en existencia, no en calidad, pero en cantidad. Eso significa que la cantidad de información es altísima, lo cual no significa que sea de utilidad, puesto que debido a su volumen es imposible de analizar para un individuo. Por ello surgen avances cómo el análisis de sentimientos, para intentar extraer información subliminal de textos de forma automatizada, es decir, sin intervención humana. Esto es parte de lo que llamamos minería de opiniones, analizar la información proporcionada por los usuarios y descifrar el significado latente de sus palabras idealmente como podría hacer una persona. Esto hace que la gran cantidad de información pueda ser también de calidad. 

Este proyecto se centrará en la obtención y el análisis de esta información que hay disponible en las redes sociales y convertir un grueso de información bruta en datos útiles que sean analizables y puedan proporcionar conclusiones prácticas para individuos o empresas. 
\cleardoublepage


\thispagestyle{empty}


\begin{center}
{\large\bfseries Technological Surveillane and Opinion Mining}\\
\end{center}
\begin{center}
Miguel Keane Cañizares\\
\end{center}

%\vspace{0.7cm}
\noindent{\textbf{Keywords}: Social Network, Twitter, Tweepy, Streaming, MeaningCloud, WordCloud, MongoDB, Pymongo, Python, Netflix, HBO}\\

\vspace{0.7cm}
\noindent{\textbf{Abstract}}\\

Nowadays social networks have become the main source of data in the world, but it's not quality information, which means that the amount of data is enourmous but that doesn't mean it's useful information. Because of its high volume it's impossible for an indivuald or even a group of individuals to analize it all. That's where Sentiment Analysis steps right in, to extract subyacent data from texts in an automated procedure without human inteference. This is what we call Opinion Mining, to analyze the information given to us by the users and decipher it's meaning as a person could do. This would make the data into quality data. 

The aim of this project is to obtain and analyze the data that's available in social network and turn a huge pile of raw data into something useful that can be analyzed and provide critical or at least practical information to indivuals or companys. 

\chapter*{}
\thispagestyle{empty}

\noindent\rule[-1ex]{\textwidth}{2pt}\\[4.5ex]

Yo, \textbf{Miguel Keane Cañizares}, alumno de la titulación Ingeniería Informática de la \textbf{Escuela Técnica Superior
de Ingenierías Informática y de Telecomunicación de la Universidad de Granada}, con DNI 76656535L, autorizo la
ubicación de la siguiente copia de mi Trabajo Fin de Grado en la biblioteca del centro para que pueda ser
consultada por las personas que lo deseen.

\vspace{6cm}

\noindent Fdo: Miguel Keane Cañizares

\vspace{2cm}

\begin{flushright}
Granada a 5 de Septiembre de 2019 .
\end{flushright}


\chapter*{}
\thispagestyle{empty}

\noindent\rule[-1ex]{\textwidth}{2pt}\\[4.5ex]

D. \textbf{Antonio Gabriel López Herrera )}, Profesor del Área de XXXX del Departamento de Ciencias de la Computación e Inteligencia Artificial de la Universidad de Granada.

\vspace{0.5cm}

%D. \textbf{Nombre Apellido1 Apellido2 (tutor2)}, Profesor del Área de XXXX del Departamento YYYY de la Universidad de Granada.


\vspace{0.5cm}

\textbf{Informan:}

\vspace{0.5cm}

Que el presente trabajo, titulado \textit{\textbf{Vigilancia tecnológica y minería de opiniones}},
ha sido realizado bajo su supervisión por \textbf{Miguel Keane Cañizares}, y autorizamos la defensa de dicho trabajo ante el tribunal
que corresponda.

\vspace{0.5cm}

Y para que conste, expiden y firman el presente informe en Granada a 5 de Septiembre de 2019 .

\vspace{1cm}

\textbf{El director:}

\vspace{5cm}

\noindent \textbf{Antonio Gabriel López Herrera }% \ \ \ \ \ Nombre Apellido1 Apellido2 (tutor2)}

\chapter*{Agradecimientos}
\thispagestyle{empty}

       \vspace{1cm}


He de agradecerle el presente a mi familia por su inestimable apoyo, a mis profesores por su profesionalidad y dedicación, a mis amigos, sin los cuales este proyecto hubiese estado terminado mucho antes y sobretodo a StackOverflow, sin el cual nada de esto hubiese sido posible. 

