%%%%%%%%%%%%%%%%%%%%%%%%%%%%%%%%%%%%%%%%%%%%%%%%%%%%%%%%%%%%%%%%%%%%%%%%
% TFG: Vigilancia Tecnológica y Minería de Opiniones en RRSS
% Escuela Técnica Superior de Ingenierías Informática y de Telecomunicación
% Realizado por: Miguel Keane Cañizares
% Contacto: miguekeca@correo.ugr.es 
%%%%%%%%%%%%%%%%%%%%%%%%%%%%%%%%%%%%%%%%%%%%%%%%%%%%%%%%%%%%%%%%%%%%%%%%




	\chapter{Introducción}
	\epigraph{"Ya no estamos en la era de la información. Estamos en la era de la gestión de la información” }{Chris Hardwick, actor}
	
		La llamada Big Data, es reina indiscutible del futuro del análisis de información. Antiguamente, el problema solía ser la falta de información disponible, pero hoy en día, el problema es que disponemos de más información de la que nadie sería capaz de procesar. Por ello, debemos automatizar dicho procesamiento, crear programas que extraigan y analicen la información a nuestro alcance para así obtener una información estadística que nos sea de utilidad, ya sea para análisis estadísticos, marketing o satisfacción del cliente. Con la información correcta se pueden tomar las decisiones correctas.
		
		A este proceso de obtención y análisis de información, lo llamaremos Minería de Opiniones. Cuya finalidad será conocer qué opina un gran número de personas sobre el tema deseado mediante lo que comparten en las redes. 
		
		
	
	\section{Motivación}
	
	Debido al auge de las redes sociales en los últimos años y los grandes cambios sociales que estas han conllevado, analizar la Big Data que nos llega de estas plataformas se ha convertido en uno de los grandes imprescindibles para todas las grandes y medianas empresas. Por ello, siendo un tema de interés y actualidad he querido trabajar en este proyecto, el cual estará centrado en obtener información de las redes y analizarla de forma que se obtenga información que pueda serle de utilidad a una empresa. Además, dentro de la aplicación, también ha sido una fuerte motivación el hecho de poder hacer este proyecto en Python, puesto que deseaba mejorar aptitudes en este lenguaje de programación. 
	
	\section{Definición del problema}
	
	La información que nos llega de las Redes Sociales (RRSS) es abrumadora, la finalidad de este proyecto será su obtención y posterior análisis.
	
	\section{Redes Sociales}
	
	Es posible distinguir entre red social y medio social, siendo la primera la interconexión de personas que se forma en la red relacionados de acuerdo a un criterio. Lo que se puede entender como medio social es la plataforma tecnológica que permite dicha interconexión en el mundo social.
	
	 Y aunque lo común sería decir que Instagram, por ejemplo, es una red social, lo cierto es que sería un medio social dentro del cual se crean múltiples redes sociales, grupos de gente con intereses comunes como puede ser un hashtag, donde la gente se junta para participar en discusiones sobre un ámbito u otro. Dentro de las plataformas vigentes, algunas de las más relevantes actualmente serían: 
	
	\subsection{Facebook}
	
	Facebook (2004) aunque no el origen (la primera red social fue SixDegrees, 2001), si es el causante de la masificación de las redes sociales en Internet. Es la red social con más usuarios en todo el mundo y dueña de las otras más cotizadas, como Instagram y WhatsApp. Esta red social fue otra de las grandes candidatas a ser objeto de la minería de opiniones de este proyecto, pero debido a su carácter privado, donde la gran mayoría de la gente tiene el perfil cerrado para que solos sus amigos puedan acceder a su contenido, suponía una dificultad insalvable a la hora de obtener un tráfico de información aceptable para el estudio. 
	
	\subsection{Twitter}
	
	Twitter (2006) fue y sigue siendo una de las redes más relevantes en la actualidad, y la que será objeto de estudio en este proyecto, debido a que es utilizada por gente de todo el mundo para la discusión de temas de actualidad, tiene un carácter público, donde los usuarios (en su mayoría) no suelen aportar apenas información personal y lo utilizan como plataforma para oír y ser escuchado en las redes. Lo cual lo hace idóneo para la minería de opiniones, pues la mayoría del contenido es escrito y público, y la propia plataforma provee a los desarrolladores de una API para poder acceder a la información desde los programas del proyecto.
	
	\subsection{Instagram}
	
	Instagram (2010) es lel medio social de moda entre los jóvenes, sus comunidades giran en torno al hashtag, los cuales son palabras precedidas por una almohadilla (\#), con las cuales los usuarios pueden encontrar un sinfín de publicaciones sobre el tópico concreto de la almohadilla. Siendo sitio preferido por los llamados influencers, los cuales son personas que debido a su alto perfil en las redes y elevado número de seguidores, poseen una cierta influencia sobre la red y pudiendo llegar incluso a generar ingresos gracias a la publicidad. Esta red casi fue la elegida para ser analizada en este proyecto, pero debido a que la mayor parte del contenido es en forma audiovisual o fotográfico, suponía una complicación añadida a la hora de minar opiniones. 
	

	


	
